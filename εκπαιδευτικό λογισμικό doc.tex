\documentclass[12pt]{article}

% Set page size and margins
\usepackage[a4paper,top=2cm,bottom=2cm,left=2cm,right=2cm,marginparwidth=1.75cm]{geometry}

% Language setting
\usepackage[english,main=greek]{babel}

% Page background & text colors
\usepackage[dvipsnames]{xcolor}
\definecolor{stronggreen}{RGB}{136, 200, 15}
\definecolor{pythonblue}{RGB}{48,105,152}
% define bg and txt colors
\definecolor{col_bg}{HTML}{FFFFFF} 
\definecolor{col_txt}{HTML}{000000}
% set the colors
\pagecolor{col_bg}
\color{col_txt}


% Useful packages
\usepackage{amsmath,amssymb}
\usepackage{amsfonts}
\usepackage[utf8]{inputenc}
\usepackage[T1]{fontenc}
\usepackage{csquotes}
\usepackage{graphicx}
\usepackage{csquotes}
\usepackage{listings}
\usepackage{algorithm}
\usepackage{algpseudocode}
\usepackage{enumitem}
\usepackage{tabularray}
\usepackage{tikz}
\usepackage{colortbl}
\usepackage{bm}
\usepackage{subcaption}

% Create plots
\usepackage{pgfplots}
\pgfplotsset{compat=newest}

% Use: \en{english text}
\newcommand{\en}[1]{\foreignlanguage{english}{#1}}

\begin{document} 

\begin{table}[ht]
    \begin{tblr}{
      @{}X[l,valign=b]X[c,valign=b]X[r,valign=b]@{}
    }

        % First line, course info
        \SetCell[c=2]{l}{Εκπαιδευτικό Λογισμικό} & & {2024-25} \\ 
        \hline
        {} & {} & {} \\

        % Title
        \SetCell[c=3]{c}{ \Large \textbf{Εγχειρίδιο Ανάλυσης και Σχεδιασμού της εφαρμογής} } \\
        {} & {} & {} \\

        % Info
        \SetCell[c=3]{c}{ \Large \it \en{\textcolor{pythonblue}{Learn Python}}} \\
        {} & {} & {} \\
        
        % Name Surname, RN
        \hline
        \SetCell[c=3]{c}{\Large \textbf{Ονοματεπώνυμα Ομάδας:} } \\
        \SetCell[c=3]{c}{\large \textbf{Μάριο Πέτο Π21138} } \\
        \SetCell[c=3]{c}{\large \textbf{Αντώνης Λάγιος Π21082} } \\
        \SetCell[c=3]{c}{\large \textbf{Ιωάννης Παπαδάκης Π21126} } \\
        \hline
        
    \end{tblr}
\end{table}
\noindent
\large \textbf{1. Ανάλυση Εφαρμογής}
\\
\par
Η εφαρμογή έχει δημιουργηθεί με στόχο την εκμάθηση της γλώσσας προγραμματισμού \en{Python} στους χρήστες. Παρέχει ύλη η οποία απευθύνεται σε αρχάριους μέχρι και προχωρημένους χρήστες.
\\
\par
Η εφαρμογή έχει χωριστεί σε δύο ενότητες, την δημιουργία ιστοσελίδων με χρήση του \en{Python} και \en{frameworks} του και την διαχείριση δεδομένων με λειτουργίες και βιβλιοθήκες του \en{Python}. Κάθε ενότητα περιλαβάνει 3 μαθήματα με υλικό που αποτελείται από κείμενα με την θεωρία, παραδείγματα κώδικα, επεξηγηματικές φωτογραφίες και βίντεο κ.α. Στο τέλος κάθε μαθήματος υπάρχει τεστ αυτοαξιολόγησης (κουίζ) για την ενίσχυση των νέων γνώσεων, επιπλέον υπάρχουν και επαναληπτικά τεστ για ολόκληρη την ενότητα εφόσον κάποιος χρήστης έχει ολοκληρώσει όλα τα τεστ αυτοαξιολόγησης σε μία ενότητα. Στην ολοκλήρωση κάθε κουίζ υπάρχει η δυνατότητα προβολής σωστών και λάθος απαντήσεων, λόγος για τον οποίο είναι λάθος μια απάντηση καθώς και ποια απάντηση είναι η σωστή. Με αυτόν τον τρόπο παρουσιάζεται μια αξιολόγηση της προόδου και ταυτόχρονα τα στατιστικά των απαντήσεων καταγράφονται στην βάση της εφαρμογής. Τέλος υπάρχει και μηχανισμός προσαρμοσμένης μάθησης στο \en{backend} που παρουσιάζει στους χρήστες εξατομικευμένα μονοπάτια μάθησης ανάλογα με τα καταγεγραμμένα στατιστικά και τον υπολογισμό των πιο αδύναμων γνώσεων.
\\
\par
Η υλοποίηση που ακολουθήθηκε επιτυγχάνει την συμβατότητα στους περισσότερους \en{browsers} με γραφικό περιβάλλον φιλικό προς στον χρήστη και ομαλή λειτουργία όλων των διαδικασιών μέσα στην εφαρμογή.
\\ \\ \\ \\
\par
\noindent
\large \textbf{2. Σχεδιασμός Εφαρμογής}
\\
\par
Η αρχιτεκτονική που ακολουθήθηκε είναι η εξής:
\begin{itemize}
    \item \textbf{\en{Backend}:} Χρησιμοποιήθηκε η \en{Firebase} της \en{Google} για την πιστοποίηση των χρηστών και την εγγραφή τους (\en{authentication}) καθώς και για την αποθηκευση δεδομένων σε μια βάση (\en{Realtime Database}). Η επικοινωνία με το \en{backend} γίνεται εξ ολοκλήρου στο \en{client-side} με \en{javascript} στον \en{browser} του χρήστη
    \item \textbf{\en{Frontend}:} Με την χρήση \en{HTML}, \en{CSS}, \en{Javascript} μέσα σε \en{ejs} αρχεία τα οποία χρησιμοποιούν \en{Node} για να τρέξουν, έχουν δημιουργηθεί όλες οι σελίδες της εφαρμογής. Το \en{CSS styling} είναι \en{custom} για τις απαιτήσεις του θέματος της εφαρμογής και τα \en{javascript} αρχεία έχουν χωριστεί ανάλογα με τον σκοπό που εξυπηρετεί το καθένα και σε ποιες σελίδες επισυνάπτονται. Επίσης χρησιμοποιέιται το \en{LocalStorage} για να αποθηκεύονται τα στοιχεία του χρήστη εφόσον είναι συνδεδεμένος ώστε να μπορεί ο αλγόριθμος να έχει πρόσβαση σε αυτά από κάθε σημείο και να σώζονται ακόμα και αν ο χρήστης κλείσει την εφαρμογή. Σημειώνεται οτι συγκεκριμένα για τα μαθήματα και τα κουίζ δεν έχουν δημιουργηθεί στατικές σελίδες για το καθένα αλλά όλα παρουσιάζονται πάνω στο ίδιο \en{sample ejs view} και τα δεδεομένα φορτώνονται δυναμικά από το κατάλληλο \en{javascript} αρχείο στα οποία σώζονται ως \en{JSON}, μειώνοντας το μέγεθος της εφαρμογής και τον χρόνο ανταπόκρισης.
    \item \textbf{\en{Database}:} Η \en{Realtime Database} της \en{Firebase} είναι μια \en{no-sql} βάση η οποία αποθηκεύει δεδομένα με \en{JSON-like} λογική. Κάθε χρήστης έχει πρόσβαση και δικαιώματα μόνο στο δικό του τμήμα δεδομένων το οποίο δημιουργείται αυτόματα με την εγγραφή του ώστε να επιτυγχάνεται η ασφάλεια των τυχόν προσωπικών του στοιχείων.
    \end{itemize}
\par
Συνοψίζοντας, οι σελίδες ή τα \en{samples} αυτών είναι \en{.ejs} αρχεία τα οποία αποτελούν τα \en{views} του \en{project} με βάση την \en{HTML} και σε κοινή λειτουργία με \en{CSS}, \en{javascript} και εικόνες πλαισιώνουν όλη την μορφή της εφαρμογής και την επικοινωνία της με το \en{backend}. Όλη η δομή περιλαμβάνεται σε ένα \en{app.js} αρχείο το οποίο επεξεργάζεται το \en{Node} για να ανεβάσει την εφαρμογή στον ιστό.
\\
\par
Ο χρήστης περιηγείται ως εξής: Πρώτον μπαίνει στο αρχικό μενού από το οποίο μπορεί να συνδεθεί ή να εγγραφεί. Ύστερα βλέπει τις διαθέσιμες ενότητες και εφόσον προϋπάρχουν στατιστικά στην βάση θα του προταθεί ποια ενότητα/μάθημα να ακολουθήσει με έξτρα βοηθητικό υλικό εφόσον αποδεχτεί, αλλιώς μπορεί να επιλέξει μια ενότητα όπου μπαίνοντας μέσα σε αυτήν μπορεί να διαλέξει ποιο μάθημα επιθυμεί να ξεκινήσει και δυναμικά να επιχειρήσει το ανάλογο κουίζ. Μπορεί επίσης εάν έχει την δυνατότητα μέσα από αυτή την σελίδα να επιχειρήσει το επαναληπτικό κουίζ όλης της ενότητας το οποίο θα αποτελείται από τα πιο αδύναμα σημεία του. Η πλοήγηση έχει στηθεί με τέτοιο τρόπο ώστε ο χρήστης να μπορεί να πλοηγηθεί σε όλες τις διαθέσιμες σελίδες όποτε το θελήσει ή ακόμη και να αποσυνδεθεί.
\\
\noindent\rule{\textwidth}{0.4pt}
\textcolor{gray}{\textit{Σημείωση: Για να τρέξετε την εφαρμογή πρέπει να εγκαταστήσετε το \en{Node.js} και ύστερα να τρεξετε σε \en{terminal} στον φάκελο της εφαρμογής \en{node app.js.}}}
\end{document}